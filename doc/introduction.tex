%
%$Id: introduction.tex,v 1.1 2003-03-10 09:42:24 gotm Exp $
%

\section{Introduction}

\subsection{What is GOTM?}
GOTM is the abbreviation for `General Ocean Turbulence Model'. It is a
one-dimensional water column model for the most important hydrodynamic
and thermodynamic processes in natural waters. The core of the model
computes solutions for the one-dimensional versions of the transport
equations of momentum, salt and heat. The key component in solving
these equations is the model for the turbulent fluxes of these
quantities. The strength of GOTM is the vast number of well-tested
turbulence models that have been implemented in the code. These models
span the range from simple prescribed expressions for the turbulent
diffusivities up to complex Reynolds-stress models with several
differential transport equations to solve. Even though, evidently, not
all turbulence models published in oceanography could be implemented,
at least one member of every relevant model family can be found in
GOTM (empirical models, energy models, two-equation models, Algebraic
Stress Models, etc).

Besides the classic combination of the hydrodynamic and turbulent part
of the model, GOTM has been growing considerably with the years, and
new parts have been developed. Sediment transport and the dynamics of
sea grass have been added, state-of-the-art numerical schemes have
been implemented, and an environment for the assimilation of data and
the computation of atmosphere-ocean interactions exists now. In
addition, there is a number of scientific research groups that adopted
GOTM for their own projects. Even though the modules developed by
these groups (biological and bio-geochemical components, air-sea
interaction modules, etc) are not part of the core structure of GOTM,
as downloadable from our web-site at {\tt www.gotm.net}, 
they are in most cases available
directly from these groups. In that sense, GOTM is an integrated,
community based software environment for an almost unlimited range of
applications in geophysical modelling.



\subsection{The idea behind GOTM}
Computer codes similar to pieces of GOTM can be found at many
scientific institutions. However, different researchers have different
goals. Some are interested in the development of turbulence models,
others in oceanic applications of these models, and yet others want to
compare the effects of different turbulence models on different
processes in the ocean or in lakes. The attempt to use one of their
specialised programs for one's own project resulted in most cases in
spending weeks of work for deciphering non-documented FORTRAN lines,
scattered with pre-historic fragments of code from more or, in most
cases, less talented programmers. Additional time had to be spend for
providing components for atmospheric forcing, etc, before the own
research project could finally be attacked.

To overcome these problems, the GOTM project was originated, its purpose
being twofold. First, GOTM should provide an integrative environment
for all researchers interested in the \emph{application} of a turbulence model
in studies of oceanic processes. Such a software should, evidently,
contain a core part for solving transport equations of mean and turbulent
quantities, but equally well routines to compute the atmosphere-ocean
fluxes from meteorological or measured data, including routines to interpolate
and manipulate them. Second, however, GOTM should also be a research tool for
those interested in the \emph{development} of turbulence models
and numerical algorithms. This implies that GOTM should always contain
the state-of-the-art models and algorithms in these disciplines. The current
version 3.0 of GOTM was developed under these premises.

In both cases, a detailed and comprehensible documentation is crucial,
and we spent a lot of effort to come close to this goal.  All methods
and models embedded in GOTM can be traced back to scientific
publications, a key requirement for the scientific use of a program.
Also, we took great care to make the FORTRAN95 code as safe, easily
understandable, and extensible, as possible.


\subsection{How to read the documentation}
This document is the official scientific documentation of GOTM,
replacing an older report by \cite{Burchardetal99}. Due to the fast
and continuing evolution of GOTM, we have been looking for a new and
flexible way of giving a comprehensive and up-to-date documentation
for GOTM. We decided for the following strategy.

Every module of GOTM is accompanied by an introductory text on the
general theory of the subject, including mathematical derivations,
bibliographic references, and the definition of the most important
variables. These introductory parts, which should give the reader a
brief theoretical overview of what is coded in the modules, are
expected to be relatively stable. References are given to more
comprehensive introductory or advanced media for each subject.

For the actual documentation of the FORTRAN95 code, which is more
likely subject to frequent changes and extensions, a different
strategy has been followed. For every {\tt module}, internal or
external {\tt subroutine} or {\tt function}, a short piece of
documentation in \LaTeX\ has been directly written into the
code. These fragments of the documentation will be updated every time
the code changes. We use a software called PROTEX, which looks into
\emph{every} FORTRAN file of GOTM, extracts the \LaTeX\ parts, and
compiles some information about the FORTRAN interfaces, public member
functions, public date members, defined parameters, etc. All these
pieces of information are assembled by PROTEX to yield a nice
documentation including table of contents, figures, tables,
references, and formulae for each part of the program. The largest
part of the report has been created in this way. Note, that PROTEX
looks for certain key words in the FORTRAN code to organise the
structure of the final document. Therefore, don't be confused if you
find things like {\tt !DESCRIPTION:}, {\tt !INTERFACE:}, {\tt !PUBLIC
DATA MEMBERS:}, etc, in the FORTRAN files. These are always preceded
by an exclamation mark, and thus invisible for your FORTRAN compiler.

If you are new to GOTM, we recommend to completely go through the core
parts of GOTM described in \sect{sec:mainIntro}, \sect{sec:meanflowIntro},
and \sect{sec:turbulenceIntro}. In these sections, you will find also
references to the relevant introductory literature. The other
parts of this documentation should be used like an encyclopedia: you
can look up things fast, when you need more information about parts
of the program. Extra comments can be found in the form of standard
FORTRAN comments, that should help users to find their way through the many
lines of the code.

A special status in this documentation has \sect{sec:cases}
illustrating the prepared scenarios for GOTM. This section gives
useful information about the theoretical background and the
implementation of each scenario currently available in GOTM. Scenarios
range from simple test cases, like a turbulent Couette flow, to full
oceanic applications including meteorological forcing and comparison
to measured data. The most simple scenarios descibed in \sect{sec:cases}
serve as a little tutorial, in which the key algorithms of
GOTM are introduced in a practical way.

This documentation does not contain information about how to download,
compile and run the code and the test cases. All information necessary
to run GOTM on a number of well-known platforms is compiled at {\tt
www.gotm.net}. If you wish to directly contact to the GOTM developers,
please write an e-mail to {\tt devel@gotm.net}. All users of GOTM,
who signed up on the GOTM web page, will be on the users' mailing
list, {\tt users@gotm.net}. Information about updates, bug fixes, and
new versions of GOTM are communicated via this list.

\subsection{Acknowledgements}

The authors of this report are grateful to the other members of the
GOTM Team for their persisting cooperation. These are particularly
those from the very first days of GOTM which took place at the Joint
Research Centre in Ispra (Italy) in 1998: Manuel Ruiz Villarreal who
worked after the Ispra time in Santiago de Compostela (Spain), Lisboa
(Portugal), Hamburg (Germany), and Warnem\"unde (Germany) before he
moved back to his home county for working in A Coru\~na
(Spain). Pierre-Phillipe Mathieu who went to Reading (U.K.) for some
time before he arrived in Frascati (Italy) recently.  We further want
to acknowledge the currently 155 subscribed users of GOTM from all
over the world who helped us to improve GOTM and motivated to go on
with this no-budget project.  It was also the important role which
GOTM played in several projects funded by the European Commission
which helped a lot to maintain GOTM.  These projects were
MAS3-CT96-0053 ('PHASE'), MAS3-CT96-0051 ('MTP II-MATER'),
MAS3-CT97-0025 ('PROVESS'), and especially CARTUM (Comparative
Analysis and Rationalisation of Second-Moment Turbulence Models), a
brainstorming activity (MAS3-CT98-0172), which brought together
turbulence specialists from all over the world.  We are finally
grateful to all those other people working on Public Domain Software
without which a project like GOTM would be unthinkable:
\LaTeX, PROTEX, LINUX and many others. 
