%
%$Id: cases.tex,v 1.10 2007-01-18 07:38:22 hb Exp $
%

\section{GOTM scenarios \label{sec:cases}}

In this section, all scenarios included in the GOTM homepage for download
are briefly discussed here. An overview is given in table 
\ref{table_scenarios}. Information about how to install and run
the scenarios can be found on the GOTM homepage, {\tt www.gotm.net}.

\vspace{10mm}

\begin{table}[h]
\begin{center}
\begin{tabular}{|l|l|l|}
\hline
Section & Title & Scenario name \\ \hline\hline 
\ref{couette} & Couette-flow & {\tt couette} \\ \hline  
\ref{channel} & Pressure-gradient driven channel flow& {\tt channel} \\ \hline
\ref{breaking_waves} & Breaking surface-waves & {\tt wave\_breaking} \\ \hline 
\ref{entrainment} & Some entrainment scenarios & {\tt entrainment} \\ \hline        
\ref{estuary} & Estarine dynamics & {\tt estuary} \\ \hline 
\ref{seagrass} & seagrass canopy dynamics & {\tt seagrass} \\ \hline 
\ref{flex} &Fladenground Experiment  & {\tt flex} \\ \hline 
\ref{nns_annual} & Annual North Sea simulation & {\tt nns\_annual} \\ \hline 
\ref{nns_seasonal} & Seasonal North Sea simulation & {\tt nns\_seasonal} \\ \hline 
\ref{liverpool_bay} &Liverpool Bay  & {\tt liverpool\_bay} \\ \hline 
\ref{gotland_deep} & Gotland Deep in Baltic Sea & {\tt gotland} \\ \hline 
%\ref{reynolds} & Middelbank in Baltic Sea & {\tt reynolds} \\ \hline 
\ref{ows_papa} &Ocean Weather Ship Papa & {\tt ows\_papa} \\ \hline 
%\ref{lago_maggiore} & Lago Maggiore & {\tt lago\_maggiore} \\ \hline 
\end{tabular}
\caption{List of GOTM scenarios described in this section}
\label{table_scenarios}
\end{center}
\end{table}


\subsection{Idealised scenarios}

In this subsection, the performance of GOTM in some idealised
turbulent flows is discussed. In these flows there are regions, where
certain analytical solutions, like the law of the wall or the Rouse
profile, apply. These solutions can be used to test the correctness of
the implementation and the accuracy of the numerical schemes. The
theoretical background is discussed in \sect{sec:turbulenceIntro} and
in the review article of \cite{UmlaufBurchard2005a}.

The first few of these idealised flows serve also as a short tutorial
for new GOTM users. We supplied several input files for these
scenarios to illustrate the performance of different turbulence models
for the same flow. It is recommended to start with the Couette-flow
described next.

\subsubsection{Couette-flow}\label{couette}

This is the simplest example designed for new users. It will tell you
about how to run a simple unstratified flow with the most frequently
used turbulence models. The term Couette-flow flow traditionally
denotes an uni-directional, unstratified, non-rotating flow confined
between two plates, of which one is moving with constant velocity. No
pressure-gradient is applied.  It is clear that this flow can also
serve as a very simple model of the steady-state flow in a
horizontally infinite ocean of finite depth, driven solely by a
shear-stress at the surface.

A set of GOTM input files (containing all specifications needed for
the runs) has been provided for 3 different turbulence models in the
sub-directories {\tt kepsilon\_nml/}, {\tt komega\_nml/} and {\tt
MellorYamada\_nml/}.  Copy all files from the subdirectory {\tt
kepsilon\_nml/} to the directory with the GOTM executable. We will
call this directory the \emph{current directory} in the following. How
to install GOTM and create the executable is described on the GOTM web
page at {\tt www.gotm.net}.  Take some time to have a look at the
contents of these files. 

In our example, the prescribed surface stress is $\tau_x= 1.027$ Pa, a
quantity that can be set in the input file {\tt airsea.nml}. This file
contains many other variables that are related to the air-sea fluxes
driving the model.

Parameters concerning the run are set in the input file {\tt
gotmrun.nml}. There, you will find for example the specification of
the water depth ($10$ m in this case) and the date and time of this
run (24 hours until a steady-state is reached). The input file
{\tt gotmrun.nml} contains mainly parameters concerning the model run,
the time step, the model time, the output format, etc.

All information about the turbulence models is read-in from the file
{\tt gotmturb.nml}. Having a look in this file, you see that we
selected {\tt tke\_method = 2} and {\tt length\_scale\_method = 8},
which corresponds exactly to the $k$-$\epsilon$ model described in
\sect{sec:dissipationeq}. The model parameters are given in the {\tt
keps} namelist. In this simple example, no Explicit Algebraic Stress
Model (see \sect{sec:secondOrder}) is solved in addition to the
transport equations for $k$ and $\epsilon$. If you compare this {\tt
gotmturb.nml} with those found in the other sub-directories (e.g.\ for
the Mellor-Yamada model) it is easy to see how different turbulence
models can be activated by changing e.g.\ the value for {\tt
length\_scale\_method}.

If you run this scenario, GOTM will write information about the run
and the turbulence model to your screen: What are the parameters of
the run, like time step, date, layers, etc? What are the model
parameters of the turbulence model? What value has the von
K{\'a}rm{\'a}n constant, $\kappa$?  What value has the decay rate in
homogeneous turbulence, $d$? And so on. All other output is written to
files called {\tt couette.out} or {\tt couette.nc}, depending on
whether you selected ASCII or NetCDF output in {\tt gotmrun.nml}.

If you analyse the results, you will find that the turbulent kinetic
energy is constant over the whole depth, whereas the profiles of the
turbulent diffusivity and the length scale are approximately
parabolic. The length scale approaches the constant slope $\kappa
\approx 0.433$ near the boundaries. If you want to change this value,
you can set {\tt compute\_kappa = .false.} in {\tt
gotmturb.nml}. Then, GOTM will automatically change the model
constants of the $k$-$\epsilon$ model to compute the value of $\kappa$
prescribed in {\tt gotmturb.nml} (see \sect{sec:analyse}).

There are other models you can use to calculate the Couette-flow.  If
you copy all files from the directory {\tt MellorYamada\_nml/} to the
current directory, GOTM will use the Mellor-Yamada model described in
\sect{sec:lengthscaleeq} with parameters set in {\tt gotmturb.nml}.  A
special role plays the so-called `generic model' described in
\sect{sec:genericeq}. Other model like the $k$-$\epsilon$ model or the
$k$-$\omega$ model by \cite{Umlaufetal2003} can be considered as
special cases of the generic model. If you copy e.g.\ the files from
{\tt komega\_nml/} to the current directory, the $k$-$\omega$ model is
run for the couette case. For this simple flow, however, model results
will be quite similiar in all cases.




\subsubsection{Pressure-gradient driven channel flow}\label{channel}

A pressure-gradient driven open channel flow is investigated here with
a prescribed surface slope $\partial_x\zeta=-10^{-5}$ at a fixed water
depth of $10$ m. No surface stress is applied, and rotation and
stratification are neglected.  The simulation is run for $24$ h until
a steady-state is reached. The specification of all these parameters
and those related to the turbulence models by use of the nml-files is
analogous to \sect{couette}.

The surface slope is set in the namelist {\tt ext\_pressure} in the
input file {\tt obs.nml}. How the parameters given in this file are
interpreted by GOTM is described in \sect{sec:extpressure} and briefly
also in {\tt obs.nml}. This file typically contains information about
``observed'' quantities that are used to either force the model (like
internal and external pressure gradients) or for comparision with
computed results. In the latter case, ``observed'' quantities are
displayed in the output file next to the computed quantities.

If you want to try out the different turbulence models mentioned in
the couette-case (see \sect{couette}), simply copy the corresponding
files from the respective subdirectories to the current directory with
the GOTM executable. Note that in {\tt gotmturb.nml} we now set {\tt
turb\_method = 3}. This implies that the turbulent fluxes are computed
from a second-order turbulence model. A new thing in GOTM 3.2 is that
parameters for the second-order model can now be directly specified
via the ``scnd'' namelist in {\tt gotmturb.nml}. For the theoretical
background of this, please see \sect{sec:secondOrder}

In the following publications some of the results in comparison to
laboratory data are shown: \cite{Burchardetal98}, \cite{Burchardetal99}, 
\cite{Burchard2002a}. The simulation has been motivated by
the work of \cite{BaumertRadach92}.



\subsubsection{Turbulence under breaking surface waves}\label{breaking_waves}

In this scenario, it is demonstrated how the effect of breaking
surface waves is parameterised in one- and two-equation models. This
is usually done by injecting turbulent kinetic energy (TKE) at the
surface, see \cite{CraigBanner94} and \cite{Craig96}. The rate of TKE
injected is proportional to the surface friction velocity cubed, as
defined in \eq{craig}. Injection of TKE at the surface leads to a thin
surface boundary layer, in which the vertical transport of TKE and its
dissipation approximately balance. This layer is sometimes called the
\emph{transport layer}. Even though there can be considerable shear in
this layer, shear-production of turbulence is negligible by definition
(also see \sect{sec:analyse}).


Different types of models are available in GOTM for the wave-breaking
scenario. The key change in {\tt gotmturb.nml} for runs with TKE
injection is to set {\tt ubc\_type = 2}, telling GOTM to set the type
of the upper boundary to TKE injection. The decay rates of the TKE and
the dissipation rate in the wave-affected layer are then an natural
outcome of the model. Note that with the KPP model, this scenario
cannot be run.

\begin{itemize}

 \item For the one-equation models, as discussed in
 \cite{CraigBanner94}, a linearly increasing macro length scale, $l$,
 is postulated with a slope of $\kappa=0.4$. This is analogous to the
 law of the wall, even though there is no physical evidence for the
 assumption that the length-scale under breaking waves behaves
 identically as in wall-bounded shear-flows. As shown by
 \cite{CraigBanner94}, an analytical solution for the one-equation
 model can be derived, but only inside the transport layer, according
 to which the TKE (and all other turbulence quantities) follows a
 power-law (see discussion in \sect{sec:generate} and
 \sect{sec:analyse}).

 If you want to simulate wave breaking with a model of this type,
 simply copy all files from {\tt prescribed\_nml/} to the current
 directoy, and run GOTM. A dynamic equation for $k$ is used, but the
 length scale is fixed, and prescribed by a triangular shape with slope
 $\kappa$ ({\tt length\_scale\_method = 2} in {\tt gotmturb.nml}, see
 \sect{sec:algebraiclength}).

 \item For two-equation models, the slope of the length scale in the
 transport layer is not simply prescribed and generally not equal to
 $\kappa$. \cite{Umlaufetal2003} generalized the solution of
 \cite{CraigBanner94} and derived analytical solutions for the
 non-linear system of equations describing the behaviour of
 two-equation models for injection of TKE at the surface. They showed
 that the TKE follows a power-law and that the length scale increases
 linearly, however, with a slope $L \neq \kappa$. They also compared
 the spatial decay of turbulence in grid stirring experiments (thought
 as an analogy to wave-breaking) to the results of several
 two-equation models.

 A numerical solution of the $k$-$\epsilon$ model can be obtained by
 copying the files in \linebreak {\tt kepspilon\_nml} to the current
 directory, and insuring that {\tt compute\_kappa = .true.}  and {\tt
 sig\_peps = .false.} in {\tt gotmturb.nml}. Because the spatial decay
 rate of the TKE is very large for this model, the wave-affected layer
 is very small, and of the order of only a few tens of centimeters for
 this scenario.  As discussed by \cite{Umlaufetal2003}, this
 disadvantage can be overcome by using the $k$-$\omega$ model with
 parameters given in {\tt gotmturb.nml} in the directory {\tt
 komega\_nml/}. The decay rates of this model nicely correspond to
 those measured in the laboratory grid strirring experiments. The
 Mellor-Yamada model has also been investigated by
 \cite{Umlaufetal2003}, but for this model, again, decay was shown to
 be too strong. In addition, the decay rate depends in an unphysical
 way on the wall-function required in this model.

 \item As an alternative to the standard $k$-$\epsilon$ model,
 \cite{Burchard2001b} suggested to make the turbulent Schmidt number
 for the $\epsilon$-equation, \eq{dissipation}, a function of the
 production-to-dissipation ratio, $P/\epsilon$. As shown in detail in
 this paper, the variable Schmidt number can be used to ``force'' the
 $k$-$\epsilon$ model to compute $\kappa$ for the slope of the length
 scale, even under breaking waves. Then, obviously, the solution of
 the $k$-$\epsilon$ model corresponds to the solution of the simpler
 one-equation model investigated by \cite{CraigBanner94}. Note again,
 however, that there is no physical evidence for $l = \kappa (z +
 z_0)$ in the wave-affected boundary layer.

 If you want to simulate wave breaking with this model, simply copy the
 files from {\tt kepspilon\_nml/} to the current directory, and make
 sure that you set {\tt compute\_kappa = .false.}  and {\tt sig\_peps
 = .true.} in {\tt gotmturb.nml}.  Results are quite similar to those
 with the prescribed length scale.

  
 \item \cite{UmlaufBurchard2003} analysed the properties of a whole
 class of two-equation models for the case of TKE injection at the
 surface. They suggested a `generic' model which could satisfy the
 power-laws under breaking waves for any desired decay rate,
 $\alpha$, and length scale slope, $L$.  This model is activated with
 the input files from {\tt generic\_nml/}.  Users can select any
 reasonable values for $\alpha$ and $L$ (and many others parameters like
 $\kappa$ and $d$), and GOTM will automatically generate a
 two-equation model with exactly the desired properties. Parameters
 are computed according to the formulae described in \sect{sec:generate}.

\end{itemize}

In all cases a surface-stress of $\tau_x= 1.027$ N\,m$^{-2}$ was
applied. After a runtime of 2 days, a steady-state with a constant
stress over the whole water column of 20 m depth is reached. The wave
affected layer can be found in the uppermost meter or so, and because
of the strong gradients in this region we used a refined grid close to
the surface. The parameters for such a `zoomed grid' can be set in the
input file {\tt gotmmean.nml} according to the decription in
\sect{sec:updategrid}.  If you want to compare the computed profiles
with the analytical solutions in \eq{power_law}, you'll need a
specification of the parameter $K$. This parameter is computed in {\tt
k\_bc()} to be found�in {\tt turbulence.F90}, where you can add a few
FORTRAN lines to write it out.

\subsubsection{Some entrainment scenarios}\label{entrainment}

This test case describes three idealised entrainment scenarios as
discussed in the review paper of \cite{UmlaufBurchard2005a}.  These
are: wind-driven entrainment into a linearly stratified fluid,
wind-driven entrainment into a two-layer fluid, and entrainment in
free convection. As in the cases before, the input files for different
turbulence closure models are contained in a number of
sub-directories.  The {\tt entrainment} test cases is also the first
test for the GOTM implementation of the KPP turbulence model described
in \sect{sec:kpp}.

For all input files, the default is a linear density stratification
due to a not necessarly linear temperature stratification (because the
equation of state is not necessarily linear).  The stratification
corresponds to $N^2=1 \cdot 10^{-4}$ s$^{-2}$. Salinity is
constant. Have look into {\tt obs.nml} to understand how different
types of initial stratifcations can be specified in GOTM. The water
depth is $H=50$ m, deep enough for the surface induced mixing not
to reach the bed within the $24$ h of simulation. Rotation is
neglected. By default, a constant wind stress of $\tau_x= 0.1027$ Pa
is set in {\tt airsea.nml}.

Note, that for all turbulence models, except the Mellor-Yamada model,
we set {\tt compute\_c3 = .true.} in {\tt gotmturb.nml}, which means
that the model constant $c_{\epsilon 3}$ in \eq{dissipation} (or its
counterpart in all other models) is computed from a prescribed
steady-state Richardson-number, $Ri_{st}$ (see discussion in the
context of \eq{Ri_st}). Some more discussion is given in
\cite{BurchardBolding2001} and \cite{UmlaufBurchard2005a}.  As pointed
out in these papers, it is the value of the steady-state Richardson
number (and thus the value of $c_{\epsilon 3}$) that determines the
mixed layer depth in almost all shear-driven entrainment scenarios.

To run the Mellor-Yamada model, use the input files in {\tt
MellorYamada\_nml/}. Looking at the results you will realize that this
model is not at all in accordance with the experimental results of
\cite{Price79} for the entrainment in a linearly stratified fluid. The
reason can be traced back to the behaviour of the turbulent length
scale in the strongly stratified thermocline.  \cite{Galperinetal88}
suggested to clip the length scale at a certain value to circumvent
this problem. Their solution can be activated by setting {\tt
length\_lim = .true.}  in {\tt gotmturb.nml}.  A second solution has
been suggested by \cite{Burchard2001c}, who computed the model
constant $E_3$ in \eq{MY} from the steady-state Richardson-number as
described above. To activate this method, select {\tt compute\_c3 =
.true.} (and {\tt length\_lim = .false.} because clipping is not
needed any more).

This scenario has been used by us in several publications as a test
for vertical mixing schemes, see \cite{Burchardetal98},
\cite{Burchardetal99}, \cite{BurchardPetersen99},
\cite{BurchardBolding2001} \cite{BurchardDeleersnijder2001},
\cite{DeleersnijderBurchard2003}, and \cite{Umlaufetal2003}.

The second entrainment scenario discussed in
\cite{UmlaufBurchard2005a} is essentially identical to the one just
described, however, it starts from a two layer stratification.  To use
this kind of initial condition, first set {\tt analyt\_method=2} in
{\tt obs.nml}, and specify the desired temperatures, {\tt t\_1} and
{\tt t\_2} for the upper and lower layer, respectively. The thickness
of the upper layer is {\tt z\_t1}. For a pure two-layer
stratification, set {\tt z\_t2=z\_t1}, otherwise you will get a linear
transition between the upper and the lower layer.

For convective entrainment, you simply need to set the momentum flux,
{\tt const\_tx}, to zero and specify an appropriate negative heat
flux, {\tt const\_heat}, in {\tt airsea.nml}, see \cite{UmlaufBurchard2005a}. 

If you run the KPP-model, some model parameters can be set in the
extra input file {\tt kpp.nml} found in {kpp\_nml/}. With this model,
the depth of the mixing layer depends mostly on the value of the
critical bulk Richardson number that can also be set in this file.
When you work with the KPP-model in free convection, don't forget to
check if the pre-processor macro {\tt NONLOCAL} is defined {\tt
cppdefs.h} (after changes in this file, don't forget to re-compile the
whole code!).  If {\tt NONLOCAL} is defined, the KPP model also
computes the non-local fluxes of heat (and salinity, if the salinity
equation is active). In any case, the $z$-coordinate of the
edges of the upper and lower mixing layers are given as {\tt zsbl} and
{\tt zbbl}, respectively, in the netCDF output file.


\subsubsection{Estuarine dynamics}\label{estuary}


In this idealised experiment, an estuarine circulation is simulated,
mainly in order to demonstrate how to use tracer advection and
internal pressure gradients in GOTM, but also to show the basic
physical process of tidal asymmetries and its impact on SPM dynamics.

The average water depth is $H=15$ m, the model is run for 10 days.
The forcing is a M$_2$ tide (of period 44714 s) which prescribes
sinusoidal time series for the vertically averaged momentum in
west-east direction with an amplitude of 1.5 m\,s$^{-1}$ and an offset
of 0.05 m\,s$^{-1}$ directed to the west in order to simulate river
run-off.  The surface elevation is sinusoidal as well with an
amplitude of 1 m and a phase shift of 1.5 hours compared to the
velocity.  A constant in time and space horizontal west-east salinity
gradient of -0.0005 ppt\,m$^{-1}$ is prescribed, advection of salinity
is turned on. In order not to obtain negative salinities, relaxation
to the initial salinity profile of 15 ppt is made.  In order to avoid
strong stratification near the surface, a small wind stress of 0.01027
N/m$^2$ is applied.

A simple suspended matter equation with constant settling velocity is
calculated in order to show the effect of tidal asymmetries on SPM
transport. It is clearly seen that during flood SPM is mixed higher up into 
the water column than during ebb, an effect which has been described 
by \cite{Geyer93}.

It is recommended to go through the description in the routines
computing the external and internal pressure gradients, see
\sect{sec:extpressure} and \sect{sec:intpressure}, to understand the
corresponding settings in the input file {\tt obs.nml}. The relaxation
scheme for salinity is described in \sect{sec:salinity}.  Essential
for this case is also the parametrisation of horizontal advection,
which is set in {\tt obs.nml} and described in
\sect{sec:salinity}. Note that horizontal advection is calculated from
the same horizontal salinity gradient that causes the internal
pressure gradient.

The result is that estuarine circulation is set on and near bed
residual velocity is directed upstream. It is interesting to have a
look into the resulting buoyancy production or Brunt-V\"ais\"al\"a
frequency.  The effect of lateral advection on stratification leads to
either production or supression of turbulence, and thus to an
asymmetric time series of the turbulent diffusivity.

For two-dimensional simulations of estuarine circulation,
see e.g.\ \cite{BurchardBaumert98} and \cite{Burchardetal2003}.


\subsubsection{Seagrass canopy dynamics}\label{seagrass}


The seagrass-current interaction has been successfully simulated
by \cite{VerduinBackhaus2000} by means of coupling an
ocean circulation model (HAMSOM) and a Lagrangian tracer model. 
The model set-up was basically two-dimensional with a vertical and
a horizontal coordinate. A harmonic swell wave travelling into the
direction of the positive $x$-coordinate had been specified at one open
boundary. 

The seagrass was represented by passive Lagrangian tracers which freely
followed the flow as long as they were located inside prescribed
excursion limits. The movement was simply frozen when the excursion limit
was reached and the flow tendency was to carry them even further out. 
Only in that situation, the seagrass tracers had an effect on the current speed
by exerting a quadratic friction on the flow. 

The basic result of \cite{VerduinBackhaus2000}
for a location inside the seagrass meadow was, that the mean kinetic energy
had a local maximum just above the upper reach of the seagrass. That was
found to be in good agreement with field measurements.  

It is interesting to perform the following two experiments:

\begin{itemize}
\item[A:] Now extra turbulence is produced by leaf-current friction,
$\alpha=0$. 
\item[B:] All friction losses between leaves and current are
converted to turbulence, 
$\alpha=1$. 
\end{itemize} 

The results for these two experiments are shown in \cite{BurchardBolding2000}.
The sensivity to $\alpha$ seems to be small, only the profiles
of averaged turbulent kinetic energy are significantly influenced. 
The results of \cite{VerduinBackhaus2000} are basically
reproduced. 
Especially, the local maximaum of mean kinetic energy just above the upper
reach of seagrass is well simulated. 
The only striking difference is that in our model the
seagrass shows an asymmetry for the excursion which is following the residual  
transport caused by the waves travelling from left to right. 
\\

{\bf Data files:}

\begin{tabular}{ll}
{\tt seagrass.dat}&  height above bed in m, excursion limit in m,
friction coefficient in m$^{-1}$.
\end{tabular}


\subsection{Shelf sea / open ocean realistic scenarios}

Here, shelf sea scenarios 
from the Irish Sea, the
North Sea and the Baltic Sea,
as well as the classical Ocean Weather Ship (OWS) Papa data 
are briefly discussed. 

\subsubsection{Fladenground Experiment ({\tt flex})}\label{flex}

A data set which has been used throughout the last 40 years
as a calibration for mixing parameterisations
has been collected during the
measurements of the Fladenground
Experiment 1976 (FLEX'76) campaign. 
Starting from a well-mixed winter situation, a surface mixed layer is
established, which is several times deepened and partly removed by
strom events.

The measurements
of meteorological forcing and potential temperature profiles
were carried out in spring 1976 in the northern North Sea at a water
depth of about 145 m and a geographical position
at 58$^{\circ}$55'N and 0$^{\circ}$32'E. 

The simulation is run from April 6 to June 7, 1976.
The \cite{Kondo75} bulk formulae have been used for pre-calculating the
surface fluxes, which are given in the data files 
{\tt momentumflux.dat} and {\tt heatflux.dat}.
Other meteorological data files are for sea surface temperature ({\tt sst.dat}) 
and short wave radiation ({\tt swr.dat}).

For further details concerning the measurements,
see \cite{Soetjeetal80} and \cite{Brockmannetal85}.
This FLEX'76 data set has been used by several authors in order to
test different mixing schemes (see e.g.\ \cite{Friedrich83},
\cite{Frey91}, \cite{BurchardBaumert95}, \cite{Pohlmann97},
\cite{BurchardPetersen99}, \cite{Mellor2001}).
\\

{\bf Data files:}

\begin{tabular}{ll}
{\tt momentumflux.dat}&  surface stress components, $\tau_x$ and $\tau_y$ in N\,m$^{-2}$ \\
{\tt heatflux.dat}    &  net surface heat flux in W\,m$^{-2}$ (negative for outgoing)\\
{\tt swr.dat}         &  short wave radiation in W\,m$^{-2}$ \\ 
{\tt sst.dat}         &  observed SST in $^{\circ}$C \\ 
{\tt pressure.dat}    & time series of surface slopes fitted to the local spring-neap cycle  \\ 
{\tt tprof.dat}       & profiles of measured potential temperature for initial conditions and \\ & validation, data are reanalysed and low pass filtered \\  
{\tt sprof.dat}       & salinity profiles from a 3D model (for initial conditions and relaxation) \\   
{\tt tprof\_ctd.dat}  & CTD-profiles of potential temperature, with some gaps \\
{\tt sprof\_ctd.dat}  & CTD-profiles of salinity, with some gaps \\
{\tt extinction.dat}  & extinction coefficients fitted to measurements \\  
\end{tabular}


\subsubsection{Annual North Sea simulation ({\tt nns\_annual})}\label{nns_annual}

Here the annual simulation of the Northern Sea at 59$^{\circ}$20' N
and 1$^{\circ}$17' E during the year 1998 as discussed
by \cite{Boldingetal2002} is performed.

For this simulation, time series of surface slopes 
$\partial_x \zeta$ and $\partial_y \zeta$ were extrapolated from observations
during autumn 1998 based on four partial tides by means of harmonic analysis.
All necessary meteorological data are from the UK Meteorological Office Model
in a 6-hourly temporal resolution ({\tt meteo.dat}).
For the period of the PROVESS campaign 
(Sep 7 - Nov 7, 1998, see \cite{Howarth_etal_2002,Boldingetal2002})  
higher resolution data (half-hourly) 
from research vessels are available and pasted
in an extended meteo data file ({\tt meteonns.dat})\footnote{See section
\ref{nns_seasonal} 
for a shorter term scenario only including the campaign period.}.
For calculating the resulting surface fluxes, the bulk formulae from
\cite{Kondo75} are used here.
For the evolution of the vertical
salinity profile, which is known to stabilise stratification
during summer months, a relaxation to results obtained with a
prognostic three-dimensional model of the North Sea 
by \cite{Pohlmann96}. 
By doing so, the horizontal advection, which is the dominant process
for salinity dynamics in the Northern North Sea, is parameterised.

The {\tt nns\_annual} scenario is also used as a basis for
driving the NPZD model (see section \ref{nns_annual_npzd}).
\\

{\bf Data files:}

\begin{tabular}{ll}
{\tt sprof.dat}         &  salinity in ppt from three-dimensional model
of \cite{Pohlmann96}\\ 
{\tt tprof.dat}         &  potential temperature in $^{\circ}$C from the three-dimensional \\
                        &  model of \cite{Pohlmann96}\\ 
{\tt pressure.dat} & sea surface slopes from tidal analysis of observations \\
{\tt meteo.dat} & meteorological data from UK Met Office model (6-hourly) \\
{\tt meteonns.dat} & meteorological data from UK Met Office model, including 
half-hourly \\ & PROVESS vampaign data for Sep - Nov 1998 \\
{\tt sst.dat} & sea surface temperature in $^{\circ}$C
from analysis by Bundesamt f\"ur \\ & Seeschifffahrt und
Hydrographie, Hamburg, Germany \\
{\tt zeta.dat} & sea surface elevation data in m from three-dimensional model
\\ & 
of \cite{Pohlmann96}
\end{tabular}


\subsubsection{Seasonal North Sea simulation ({\tt nns\_seasonal})}\label{nns_seasonal}

This Northern North Sea seasonal experiment has been carried out in the 
framework
of the PROVESS (PROcesses of VErtical mixing in Shealf Seas) project 
(MAS3-CT97-0025, 1998-2001) which has been funded by the European Communities 
MAST-III program.

The observations in the Northern North Sea were carried out in September 
and October 1998.
Here, a period of 20 days from October 7 - 27, 1998 is simulated.
All forcing and validation data have been carefully processed from
observations during this PROVESS-NNS experiment. 

Two different observational dissipation rate data sets are included:

\begin{tabular}{ll}
{\tt eps\_fly.dat}& data from a FLY profiler, processed by School
of Ocean Sciences, \\ & University of Bangor, Wales \\  
{\tt eps\_mst.dat}& data from an MST profiler, processed by JRC, Ispra, Italy. 
\end{tabular}
\\

These files can be read in into GOTM through the {\tt eobs} namelist in
{\tt obs.nml}.  The dissipation rate has only been observed at 
short intervals, periods without observations are set to minimum 
values in the files. These dissipation rate observations are read in
into GOTM in order to allow for proper interpolation to the temporal 
and spatial
output steps, and they are not used for any type of nudging.

The data files are prepared such that the maximum simulation
interval can be extended to September 7 at 10.00 h -- 
November 2 at 13.00 h, 1998. 

For details on dissipation rate data processing, see \cite{Prandkeetal2000}.

For discussions of various model simulations, see \cite{Burchardetal2002}
and also the annual simulation in section \ref{nns_annual} and
\cite{Boldingetal2002}.
\\ 

{\bf Other data files:}

\begin{tabular}{ll}
{\tt sprof.dat} &  salinity in ppt from CTDs and microstructure \\ & 
shear probes from several ships\\
{\tt tprof.dat} &  potential temperature in $^{\circ}$C from CTDs and microstructure \\ &
shear probes from several ships\\
{\tt pressure.dat} & sea surface elevation gradients analysed from \\
                   & a triangle of pressure gauges \\
{\tt w\_adv.dat} & vertical velocities analysed from vertical displacement 
\\ & of pycnocline \\
{\tt velprof.dat} &  horizontal velocities from bottom mounted ADCP \\
{\tt meteo\_dana.dat} & meteorological observations from R/V Dana, only used 
\\ & for {\tt calc\_fluxes=.true.}\\
\end{tabular}


\subsubsection{Liverpool Bay ({\tt liverpool\_bay})}\label{liverpool_bay}

The observations for this scenario
have been carried out by \cite{Rippethetal2001A} 
in the Liverpool Bay ROFI \index{region of  
freshwater inflow} on July 5 and 6, 1999 at a position
of 53$^{\circ}$28.4'N, 3$^{\circ}$39.2'W.
This period is about three days after spring tide,
with calm weather and clear sky.
The dissipation rate measurements were carried out with a FLY shear probe
mounted on a free-falling profiler. 
Sensors for temperature and conductivity attached to the profiler
give detailed information on the vertical density distribution
during each cast.
Nearby, an ADCP was mounted on the bottom, giving information on the
vertical velocity structure. Some accompanying CTD casts were made in
order to achieve estimates for the horizontal gradients of
temperature and salinity.   
For further details concerning the observations, see
\cite{Rippethetal2001A}. 

The surface fluxes are based on ship observations and from a
nearby meteorological station at Hawarden. From the ship,
wind speed  and direction at 10 m above the sea surface and
air pressure have been taken. From Hawarden station, observations of
dry air,
wet bulb and dew point temperature are used. Since the
surface fluxes
are calculated externally by means of 
bulk
formulae of \cite{Kondo75}, the sea surface temperature
from measurements (FLY profiler) 
has been used.
The bed roughness has been estimated from near-bed ADCP measurements
as $z_0^b\approx 0.0025$ m 
by means of fits to the law of the wall.
The external pressure gradient due to surface slopes is
estimated according to a method suggested by \cite{Burchard99}
by means of adjustment to near bed velocity observations.
The CTD casts carried out during the campaign did only allow for rough
estimates of the horizontal density gradient. 
The  
horizontal salinity and temperature gradients for a typical summer situation
have been estimated
by \cite{Sharples92} to $\partial_s S=0.0425$ ppt\,km$^{-1}$ and
$\partial_s T=-0.0575$ K\,km$^{-1}$, respectively. Here, $s$ is the
gradient into the direction $\alpha=78^{\circ}$ rotated
anti-clockwise from North.
\\ 

{\bf Data files:}

\begin{tabular}{ll}
{\tt sprof.dat} & salinity in ppt from free-falling shear-probe \\
{\tt tprof.dat} & potential temperature in $^{\circ}$ from free-falling shear-probe \\
{\tt pressure.dat} & near-bed velocity from ADCP for external pressure forcing \\
{\tt zeta.dat} & sea surface elevation from pressure gauge \\
{\tt velprof.dat} &  horizontal velocities from bottom mounted ADCP \\
{\tt eprof.dat} & observed dissipation rates in W\,kg$^{-1}$ \\
{\tt heatflux.dat} & surface heat fluxes calculated accroding to \cite{Kondo75} \\
{\tt momentumflux.dat} & surface momentum fluxes calculated according to \cite{Kondo75} \\
{\tt swr.dat} & short wave radiation data \\
\end{tabular}
\\


The numerical simulations of this scenario
has been described in \cite{Simpsonetal2002}.

\subsubsection{Gotland Deep ({\tt gotland})}\label{gotland_deep}

These simulations are made for the location of
station 271 Central Eastern
Gotland Sea of the Baltic Sea at
$20\,$E and $57.3\,$N with a water depth of about $250\;$m.
Initial conditions for temperature and salinity are derived from measurements. 
Meteorological
forcing was available from the ERA15 reanalysis data set
(http://wms.ecmwf.int/research/era/Era-15.html). 
For the penetration of solar radiation into the
water column, fairly turbid water (Jerlov type IB) is assumed. 
Salinity concentrations are nudged to observations with a time scale of
$\tau_R=$ 2 days.

For the comparison of simulated temperature and salinity
and observations  we have used mainly
data from the COMBINE program. All environmental monitoring within HELCOM
and the Baltic marine environment is carried out under the COMBINE program.
The COMBINE program runs under the umbrella of HELCOM.
HELCOM is the governing body of the {\it Convention on the Protection of the
Marine Environment of the Baltic Sea Area} - more usually known as the
Helsinki Commission ({\tt www.helcom.fi}). In a regular schedule data from
stations in the Baltic Sea are collected. Parts of these data are maintained
inter alia at the Baltic Sea Research Institute Warnem\"unde
and can be used for
scientific work. 

Model results and observations are compared for the years 1994-1996.
For the discretisation, the water column has been
divided into 100 vertical layers, with a strong zooming towards the surface,
resulting in a mean near-surface resolution of less than 0.5 m.
The time step for these simulations is set to $\Delta t=600$ s.
\\

{\bf Data files:}

\begin{tabular}{ll}
{\tt meteo.dat} &  meteorological data extracted from the ERA15 reanalysis 
                   data set  \\
{\tt precip.dat} &  precipitation 
ata extracted from the ERA15 reanalysis 
                   data set  \\
{\tt sprof\_271.dat} & deep salinity profiles at station 271\\ 
{\tt tprof\_271.dat} &  deep temperature profiles at station 271\\
\end{tabular}
\\

The meteorological data have been compiled by Frank Janssen 
(IOW, Baltic Sea Research Institute Warnem\"unde, Germany), and
the temperature and salinity profiles have been collected from the IOW
data bank by Iris Theil (University of Hamburg, Germany).

These data have been used for simulating the Gotland Deep ecosystem dynamics
for the years 1983-1991, see \cite{Burchardetal05},
see also section \ref{gotland_iow}.

%\subsubsection{Middelbank}\label{reynolds}
%
%Here a campaign (REYNOLDS, funded by the German Federal Ministry for
%Education and Research, chief-scientist Hans Ulrich Lass, IOW)
%in the Eastern Bornholm Basin
%(55$^{\circ}$ 35' N, 16$^{\circ}$ 39' E, mean water depth: 55 m)
%is simulated. The simulation period is
%August 30, 2001 at 17 h to September 9, 2001 at 14 h.
%The water column is characterised by a thermocline at about 25 m depth 
%and a halocline at about 50 m depth. The simulation period 
%is charaterised by storms up to 0.2 N\,m$^{-2}$.
%As forcing, surface stress, heat fluxes and solar radiation
%has been calculated on the basis of meteorological observations
%according to \cite{Kondo75}.
%The barotropic pressure gradient has been recalculated 
%from vertically averaged observed velocity profiles, see section
%\ref{sec:meanflowIntroPhysics}.
%As initial conditions, observed temperature, salinity and velocity profiles
%are used. 
%Additionally the vertical velocity at the thermocline has been diagnosed from
%temperature observations and is used for vertical advection,
%see section \ref{sec:meanflowIntroPhysics}.
%The turbulent dissipation rate $\eps$ has been observed during two 
%sub-periods, such that turbulence model results may be compared with 
%observations. 
%\\
%
%{\bf Data files:}
%
%\begin{tabular}{ll}
%{\tt eprof.dat} & profiles of observed dissipation rate in W\,kg$^{-1}$ \\
%{\tt heatflux.dat} & surface heat flux and solar radiation in  W\,m$^{-2}$ \\
%{\tt momentumflux.dat} & surface momentum flux in  N\,m$^{-2}$ \\
%{\tt pressure.dat} & vertically averaged velocity components in m\,s$^{-1}$ \\
%{\tt sprof.dat} & profiles of observed salinity in psu \\
%{\tt sss.dat} & time series of sea surface salinity in psu \\
%{\tt sst.dat} & time series of sea surface temperature in $^{\circ}$C \\
%{\tt tprof.dat} & profiles of observed temperature in $^{\circ}$C \\
%{\tt velprof.dat} & profiles of observed velocity components in m\,s$^{-1}$ \\
%{\tt vertvel.dat} & profiles of diagnosed vertical velocity at thermocline 
%                    depth in m\,s$^{-1}$ \\
%\end{tabular}
%\\
%
%So far, these data have not yet been published. 

%\subsection{Open ocean scenarios}
%
%The two open ocean scenarios introduced here are two classical
%test cases from the Northern Pacific Ocean. For an overview, see
%\cite{Martin85}.

\subsubsection{Ocean Weather Ship Papa ({\tt ows\_papa})}\label{ows_papa}

This scenario is a classical scenario 
for the Northern Pacific, for which
long term observations of meteorological parameters
and temperature profiles are available. The station Papa at 145$^{\circ}$W,
50$^{\circ}$N has the advantage
that it is situated in a region where the horizontal advection of
heat and salt is assumed to be small. Various authors used these data for
validating turbulence closure schemes (\cite{Denman73}, \cite{Martin85},
\cite{Gasparetal90}, \cite{Largeetal94}, \cite{KanthaClayson94},
\cite{dAlessioetal98}, \cite{Burchardetal99}, \cite{Villarreal2000},
\cite{AxellLiungman2001}, \cite{BurchardBolding2001}).

The way how bulk formulae for the surface momentum and heat fluxes
have been used here is discussed in detail in \cite{Burchardetal99}.

For mixing below the thermocline, an 
internal wave and shear instability
parameterisation 
as suggested by \cite{Largeetal94} has been used.
The maximum simulation time allowed by
the included surface forcing file and the temperature profile file
is January 1 (17.00 h), 1960 - December 31 (12.00 h), 1968.
In this scenario, the simulation time is run from
March 25, 1961 (0.00 h) to March 25, 1962 (0.00 h).
\\

{\bf Data files:}

\begin{tabular}{ll}
{\tt sprof.dat} &   salinity profiles in ppt of monthly climatology from 
Levitus data set. \\ & First profile interpolated to January, 1st \\ 
{\tt tprof.dat} & profiles of measured potential temperature for 
initial conditions and \\ & relaxation \\    
{\tt heatflux.dat} & surface heat fluxes calculated according to
\cite{Kondo75} \\
{\tt momentumflux.dat} & surface momentum fluxes calculated according to \cite{Kondo75} \\
{\tt swr.dat} & short wave radiation data \\
\end{tabular}
\\

This scenario has been discussed in detail by \cite{Burchardetal99}.
We are grateful to Paul Martin for providing the meteorological data
and the temperature profiles, see also \cite{Martin85}.

%\subsection{Lake scenarios}
%
%So far, the Lago Maggiore scenario discussed in section \ref{lago_maggiore}
%is the only lake scenario. 
%
%\subsubsection{Lago Maggiore}\label{lago_maggiore}
%
%The measurements for this Lago Maggiore
%scenario were made during three days in winter 1995
%(December 18-21) at the shore of Ispra 
%(45$^{\circ}$ 49,244'N, 8$^{\circ}$ 36,377'E). 
%The measurements were carried out with an
%uprising profiler located 150 m from the shore at a water depth of 42 m. 
%Such the sampled depth interval ranged from 30 m up to the surface.
%On the profiler, an MST shear 
%probe, a fast temperature sensor
%and temperature and conductivity probes were mounted such that
%profiles of turbulent dissipation rate $\epsilon$, temperature variance
%$\epsilon_\theta$, mean temperature $\theta$ and mean salinity $S$
%could be derived. For a detailed description of the data analysis, see
%\cite{Stipsetal2002}.  
%
%Wind speed was measured from a small buoy about 30 m away from the 
%probe location with an anemometer at a height of 95 cm above the water 
%surface. The accuracy is $\pm 0.1$ m\,s$^{-1}$. 
%Air temperature and relative humidity were recorded at the measurement 
%location on shore at a height of 10 m above lake surface.
%The cloud cover has been estimated every hour. 
%Incident solar radiation was measured at the meteorological station in 
%Pallanza, in a distance of about 10 km from the 
%measuring site.
%An analysis of heat fluxes obtained by various bulk 
%formulae showed
%however a significant deviation between the heat content of the water column
%and accumulation of these heat fluxes. This could be due to the fact that these
%bulk formulae are designed for oceanic conditions such that they are not
%valid for a lake with weak wind conditions. Thus, instead of using the
%calculated surface heat fluxes from bulk 
%formulae, they were
%calculated from the heat gain of the water column under consideration
%of the solar radiation.
%\\
%
%{\bf Data files:}
%
%\begin{tabular}{ll}
%{\tt salz\_lmd95.dat} &  profiles of measured salinity in ppt for initial 
%conditions and relaxation \\
%{\tt temp\_lmd95.dat} & profiles of measured potential temperature for 
%initial conditions and \\ & relaxation \\    
%{\tt eps\_lmd95.dat} & profiles of measured dissipation rate for validation \\
%{\tt hflu2\_05lt.dat} & surface heat fluxes calculated according to \cite{Kondo75} \\
%{\tt momentumflux.dat} & surface momentum fluxes calculated according to \cite{Kondo75} \\
%\end{tabular}
%\\
%
%For a discussion of the simulation, see \cite{Stipsetal2002}. 

\subsection{Biogeochemical scenarios}

The functionality of biogeochemical models in GOTM is demonstrated here for
four different scenarios:

\begin{itemize}
\item The idealised channel flow scenario introduced in
section \ref{channel} is here complemented with a simple suspended matter
equation, see section \ref{channel_rouse}.
\item The PROVESS Northern North Sea scenario (see section \ref{nns_annual})
is here coupled to the NPZD model, see section \ref{nns_annual_npzd}.
\item The Fladenground Experiment (FLEX'76), see section \ref{flex}
is here coupled to the \cite{Fashametal1990} biogeochemical model,
see section \ref{flex_fasham}.
\item The multiannual Gotland Deep scenario (see section \ref{gotland_deep}) 
is here
coupled to the biogeochemical model by \cite{Neumannetal2002},
with adaptations by \cite{Burchardetal05}, see section \ref{gotland_iow}.
\end{itemize}

\subsubsection{Channel flow - Rouse profile}\label{channel_rouse}


In this scenario, the water depth and the surface slope have been
chosen identical to those in the purely physical
{\tt channel} scenario introduced in section \ref{channel}.

Under certain conditions, the suspended matter equation 

\begin{equation}\label{eq_rouse}
\partial_t C + \partial_z\left(w_c C - \nu_t \partial_z C\right) = 0,
\end{equation}

has an analytical solution:

\begin{itemize}
\item Constant settling velocity $w_c$
\item Parabolic eddy diffusivity $\nu_t$
\item Reflective bottom and surface
\item Steady-state solution
\item No sources and sinks
\end{itemize}

Let the eddy diffusivity profile be parabolic with

\begin{equation}
\nu_t = \kappa u_* (-z)\frac{D+z_0+z}{D+z_0}
\end{equation}

with the depth $D$, the bottom roughness length $z_0$, the van Karman
number $\kappa$ and the bottom friction velocity $u_*$ and the vertical
coordinate $z$.
Then the analytical solution of (\ref{eq_rouse}) is

\begin{equation}\label{rouse_analytical}
\displaystyle
\frac{C}{C_0} = \left(\frac{-z}{D+z_0+z} \right)^{-w_c/(\kappa u_*)},
\end{equation}

where $C_0$ is the suspended matter concentration at $z=-(D+z_0)/2$i
and depends on the initial conditions for $C$.
The Rouse number is then defined as:

\begin{equation}
R =\frac{-w_c}{u_*}.
\end{equation}

When running the {\tt rouse} scenario with a two-equation turbulence 
closure model, then the analytical solution for the Rouse profile 
is only approximated, since the eddy diffusivity deviates from 
(\ref{rouse_analytical}).
In order to be closer to the analytical solution, it is ncessary to chose in
{\tt gotmturb.nml} the analytical parabolic profile for eddy diffusivity,
which is done by the follwing settings in {\tt gotmturb.nml}:

\begin{verbatim}
 &turbulence
  turb_method=     2,
  tke_method=      1,
  len_scale_method=4,
  stab_method=     1
\end{verbatim}

This Rouse scenario may be calculated with Eulerian concentrations 
or with Lagrangian particles, depending on the setting
of {\tt bio\_eulerian} in {\tt bio.nml}.
When using the Lagrangian particle method, it is advisable
to average the concentration over all time steps 
which belong to one output time step, 
by setting {\tt bio\_lagrange\_mean=.true.}

It is also possible to include a sink term at the bed
(for simulating the effect of grazing by benthic filter feeder),
the seetings for this have to be made in {\tt mussels.nml}.


\subsubsection{Northern North Sea - NPZD model}\label{nns_annual_npzd}

In order to provide a typical and clearly defined physical environment
for testing the NPZD model discussed by \cite{Burchardetal2005b},
we use an annual simulation of the water column in the Northern North
Sea. The setup is similar to the one described in section
\ref{nns_annual} and validated
in detail by \cite{Boldingetal2002}.
The photosynthetically available radiation (PAR)
has been simply taken as visible part of the short-wave radiation,
a feedback of increased turbidity due to plankton blooms to the
light absorption has not been considered for the heat budget.
For further details of the physical model setup, see section \ref{nns_annual}.

As shown by \cite{Burchardetal2005b}, the explicit ODE solvers
introduced in section \ref{sec:ode-solver} do not guarantee
non-negative solutions for biogeochemical concentrations when the
biogeochemical model is stiff, i.e.\ when time scales are involved which may
be shorter than the time step. In order to make the NPZD model stiff,
\cite{Burchardetal2005b} chose a
half-saturation nutrient concentration $\alpha$ 
of 0.02  mmol N\, m$^{-3}$, whereas the typical values for $\alpha$ would be 
between 0.2 and 1.5 mmol N\, m$^{-3}$.
This has the consequence that nutrient is taken up by phytoplankton
even at low concentrations, which strongly decreases the
time scale of this process. The overall phytoplankton evolution over
an annual cycle 
is not much affected by this
manipulation, except from the fact that now the summer surface nutrient
concentrations are much lower.
It should be noted that such low half saturation concentrations
for nutrients have actually been observed in the oceanic mixed
layer. \cite{Harrisonetal1996} calculated for the mixed layer
of the North Atlantic mean half saturation concentrations
for nitrate and ammonium as small as 0.02  mmol N\, m$^{-3}$.

In order to demonstrate the advantages of the 
Modified Patankar schemes over the fully explicit schemes, 
the simulation carried out here is based on
a time step of 2 h for the physical part. For the biogeochemical part,
a time splitting is used (set {\tt split\_factor}  in
{\tt bio.nml}) such that fractional time
steps are possible for the biogeochemical part.
Thus, by using a time step of $\Delta^{phy} t = 7200$ s for the
physical part and iterating the biogeochemical part 1,4 and 36 times
per physical time step with unchanged physical forcing,
biogochemical time steps of  $\Delta^{bio} t = 7200$ s,
$\Delta^{bio} t = 1800$ s and $\Delta^{bio} t = 200$ s,
respectively, can be obtained.
By doing so, it is possible to use exactly the same physical forcing
for all ODE solvers and all biogeochemical time steps.

When using the explicit ODE solver (to do so, set {\tt ode\_method}
in {\tt bio.nml} to 1, 2 or 3), the summer nutrient concentration
goes down to negative values.
This does not happen for the other conservative methods
7 and 8 (first- and second-order Modified Patankar schemes) and
10 and 11 (first- and second-oder Extended Modified Patankar
schemes). 


\subsubsection{Fladenground Experiment - \cite{Fashametal1990} model
}\label{flex_fasham}

During the Fladenground Experiment 1976, extensive biogeochemical
observations have been carried out as well. Here,
the modelling work of \cite{KuehnRadach1997} will be
reproduced with GOTM. All GOTM modelling details have been
documented by \cite{Burchardetal05}.

The bloom at the Fladenground started with the onset of thermal 
stratification of the water
column at 19 April (Julian day 110), reached its maximum at 1 May
with an depth-integrated phytoplankton biomass of about 11 g C m$^{-2}$
(\cite{Radachetal1980}). At 16 May the phytoplankton stock reached again the
pre-bloom level. This main bloom was dominated by diatoms;
flagellates constituted a smaller secondary bloom some weeks later.
In accordance with the rapid production of organic matter the
nitrate pool was depleted: from 8 mmol N m$^{-3}$ before the bloom to
less than 0.1 mmol N m$^{-3}$ at the end of the bloom
(\cite{Brockmannetal1983}).
The observed average daily primary production during the bloom was
1.2 g C m$^{-2}$ d$^{-1}$ (\cite{Weigeletal1980}).
The part of organic substance settling to the bottom was estimated
from sediment trap measurements to about $20 \pm 10$ \%
of the primary production during that period
(\cite{Daviesetal1984}, \cite{Radachetal1984}). The remaining 80 \% were
obviously used by the zooplankton and partly --
via dissolution of the PON -- also by bacteria. From estimates
of the grazing pressure of the dominant copepod, {\it Calanus finmarchicus}
(see \cite{Krauseetal1980}) it was concluded that additionally herbivorous
microzooplankton must have played an essential role in grazing
on the phytoplankton stock.
The role of advection on the ecosystem dynamics was discussed
by \cite{Eberleinetal1980}. Thus the total FLEX '76 period could be
divided into two phases: the first 6 weeks (until 6 May) with
only weak horizontal advection, and the following 4 weeks with distinctively
larger influence of advection on the nutrient concentrations.

This biogeochemical scenario has been reproduced by \cite{KuehnRadach1997}
by means of the \cite{Fashametal1990} biogeochemical
model together with a
one-equation turbulence closure model. \cite{Burchardetal05},
from which most of the present section has been adapted
documented the implementation of this model into GOTM.

\subsubsection{Gotland Deep - \cite{Neumannetal2002} model}\label{gotland_iow}

The Gotland Deep scenario (Central Baltic Sea) which has been described 
in section \ref{gotland_deep} has been used for 
demonstrating the implementation of the \cite{Neumannetal2002}
biogeochemical model into GOTM, see \cite{Burchardetal05}.
For the description of the implementation into GOTM,
see \cite{Burchardetal05}. 

Some adaptations to the \cite{Neumannetal2002} had to be made,
in order to provide acceptable results for the
ecosystem simulations in the Gotland Deep. The 
the maximum growth rate of diatoms from
a value of $r_1^{\max}=1.5$ d$^{-1}$ to $r_1^{\max}=2.0$ d$^{-1}$.
This is due to the fact that the simple turbulence closure scheme
(\cite{Pacanowski81})
used by \cite{Neumannetal2002} mixed substantially less than the
$k$-$\eps$ used here.
The surface
fluxes of nutrients have been calibrated in such a way that
winter nutrient concentrations are close to observations.
By doing do, the effect of lateral nutrient transport is
parameterised.

For further details, see \cite{Burchardetal05}.
