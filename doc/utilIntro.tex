%
%$Id: utilIntro.tex,v 1.1 2003-03-10 09:42:25 gotm Exp $
%

\section{Utilities}\label{sec:util}

\subsection{Introduction}

Here, some completely different utility modules and
routines are assembled,
such as the time module (see {\tt time.F90}) keeping track of all 
time calculations,
the tridiagonal module with the Gaussian solver for systems of equations
with tri--diagonal matrices (see {\tt tridiagonal.F90}),
the equation of state module (see {\tt eqstate.F90}) and some 
helpful routines such as the general advection--diffusion solver
(see {\tt yevol.F90}), the vertical advection routine
(see {\tt w\_split\_it\_adv.F90}) and the vertical interpolation routine
(see {\tt gridinterpol.F90}). Some one--dimensional test calculations for
the vertical advection schemes included in
{\tt w\_split\_it\_adv.F90} (see section \ref{sec:advection})
are shown in figure \ref{OneDAdvection}.

\begin{figure}[h]
\begin{center}
\psfrag{Y}[cc][][0.8]{$\Psi$}
\psfrag{Analytical}[cc][][0.4]{Analytical}
\psfrag{Numerical}[cc][][0.4]{Equidistant}
\psfrag{Equidistant}[cc][][0.4]{Equidistant}
\psfrag{Non--equidistant}[cc][][0.4]{Non--equidistant}
\psfrag{X}[ct][][0.8]{Unlimited P$_2$ scheme}
\includegraphics[width=6.5cm,bbllx=50,bblly=50,bburx=543,bbury=396]{./figures/cubenoneq.ps}
\psfrag{X}[ct][][0.8]{Limited P$_2$--PDM scheme}
\includegraphics[width=6.5cm,bbllx=50,bblly=50,bburx=543,bbury=396]{./figures/rectUQ.ps}
\psfrag{X}[ct][][0.8]{Limited MUSCL scheme}
\includegraphics[width=6.5cm,bbllx=50,bblly=50,bburx=543,bbury=396]{./figures/rectmuscl.ps}
\psfrag{X}[ct][][0.8]{Limited Superbee scheme}
\includegraphics[width=6.5cm,bbllx=50,bblly=50,bburx=543,bbury=396]{./figures/rectsuper.ps}
\caption{
One--dimensional advection scheme test showing the analytical solution
(dashed line) and the numerical approximation (contineous lines) after
6 revolutions through a periodic domain of length
$L$ = 100 m with $\Delta x$ = 1 m and an advection speed of
$u=1$ m\,s$^{-1}$. The Courant number was $c=0.5$ for the equidistant 
grid--spacing
and the maximum Courant number was $c=0.88$ 
for the non--equidistant grid--spacing.
Four different schemes have been used. It should be noted for the
Superbee scheme (lower right) that it is anti-diffusive
in the sense that it sharpens smooth gradients.
For details concerning the used schemes, see \cite{Pietrzak98}
and \cite{BurchardBolding2002}.
}\label{OneDAdvection}
\end{center}
\end{figure}

